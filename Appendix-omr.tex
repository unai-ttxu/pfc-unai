\chapter{OMELETTE Mashup Registry (OMR)}

\section{Installation of Sesame with uSeekM (+PostgreSQL +PostGIS)}

Requirements:

\begin{itemize}

\item A Java Servlet Container that supports Java Servlet API 2.4 and Java Server Pages
(JSP) 2.0, or newer. We recommend using a recent, stable version of Apache Tomcat.
(http://tomcat.apache.org/)
\item A recent, stable version of PostgreSQL Server (32Bit).

(http://www.postgresql.org/download/windows/)
\item PostGIS extension for PostgreSQL. (Can be installed with the Stack Builder that
   comes with PostgreSQL.)
\item Extended Sesame HTTP Server with Indexing - USeekM

   (https://dev.opensahara.com/projects/useekm/wiki/HttpServer)
Installation steps:
\item First install the PostgreSQL Server for Windows 32Bit (32Bit because the PostGIS
   extension is not yet 100\% compatible with the 64Bit server and thus is not available in
  the Stack Builder Installer).
\item After installing the PostgreSQL Server start the Stack Builder and select PostGIS 1.x
   under "Spatial Extensions" and install it to the running PostgreSQL instance.
\item Start the pgAdmin tool to create a database user and database for your index.
\item Create a new Login-Role (useekm is used as name and password in the example)
   with superuser rights.
\item Create a new database (useekm is used as name in the example). Select the created
   user as owner. Choose "template\_postgis" as template (IMPORTANT!).
\item Your database should be good to go.
\item Install Apache Tomcat and get it running. There should be no configuration needed.
\item Download the latest useekm-http-server and useekm-http-workbench *.war files from
   https://dev.opensahara.com/nexus/content/repositories/releases/com/opensahara/
\item You may rename them to openrdf-sesame.war and openrdf-workbench.war if you
   want to replace your current sesame installation.
\item Put them in the Tomcat webapps folder. They should automatically get deployed.
\item Your extended sesame server should be good to go.
\item Create a configuration file for your indexed repository like this.
\item The example indexes the http://purl.org/dc/elements/1.1/description predicate.
\item You may need to edit the database connection settings, like username, password and
   the URL with the database name.
\end{itemize}

\begin{lstlisting}[style=consola]
<beans xmlns="http://www.springframework.org/schema/beans"
xmlns:xsi="http://www.w3.org/2001/XMLSchema-instance"
xsi:schemaLocation="http://www.springframework.org/schema/beans
http://www.springframework.org/schema/beans/spring-beans-3.0.xsd">
<!-- The id "repository" is mandatory! -->
<bean id="repository" class="org.openrdf.repository.sail.SailRepository">
<constructor-arg>
<bean class="com.useekm.indexing.IndexingSail">
<constructor-arg ref="sail" />
<constructor-arg ref="indexerSettings" />
</bean>
</constructor-arg>
</bean>
<!-- This example uses the NativeStore as the underlying sail, you could also use the
MemoryStore -->
<bean id="sail" class="org.openrdf.sail.nativerdf.NativeStore" />
<!-- Please customize the indexer settings: -->
<bean id="indexerSettings" lazy-init="true"
class="com.useekm.indexing.postgis.PostgisIndexerSettings">
<property name="defaultSearchConfig" value="simple" />
<property name="dataSource" ref="pgDatasource" />
<property name="matchers">
<list>
<bean class="com.useekm.indexing.postgis.PostgisIndexMatcher">
<property name="predicate"
value="http://purl.org/dc/elements/1.1/description" />
<property name="searchConfig" value="simple" />
</bean>
</list>
</property>
<property name="partitions">
<list>
<bean class="com.useekm.indexing.postgis.PartitionDef">
<property name="name" value="description" />
<property name="predicates">
<list>
<value>http://purl.org/dc/elements/1.1/descr
iption</value>
</list>
</property>
</bean>
</list>
</property>
<!-- You can add additional configuration, such as index partitions to optimize
performance. See the documentation. -->
</bean>
<bean id="pgDatasource" lazy-init="true"
class="org.apache.commons.dbcp.BasicDataSource" destroy-method="close">
<property name="driverClassName" value="org.postgresql.Driver"/>
<property name="url" value="jdbc:postgresql://localhost:5432/useekm"/> <!--
CUSTOMIZE! -->
<property name="username" value="useekm"/>
<!--
CUSTOMIZE! -->
<property name="password" value="useekm"/>
<!--
CUSTOMIZE! -->
</bean>
</beans>
\end{lstlisting}

\begin{itemize}
\item Save this as configuration.xml to a folder in which the webapps have access to.
\item Open the URL to the workbench http://localhost:8080/useekm-http-workbench).
\item Enter the URL to your Sesame server. (ex.: http://localhost:8080/useekm-http-server).
\item Create a new repository.
\item Choose USeekM Store as Type.
\item Choose a name and ID.
\item Press Next and enter the full path to the configuration file mentioned above. (use
   slashes instead of backslashes!).
\item Press create and the repository and the tables in the database should get created and
initialised.
\end{itemize}

Now you have the prerequisites to install the DataGridService for hosting your own
OMELETTE Mashup Registry.

\section{Installation of the DataGridService}
We created an installable package for Windows to run a stand-alone DataGridService with all
current features of the OMR and a RESTClient with a GUI
(https://vsr.informatik.tu-chemnitz.de/demo/omr/omrsetup.zip). Just install the package and
two folders will be created in the selected folder - one with the RESTClient and one with the
DGS. Before starting the DataGridService you may need to configure it to connect to the
correct Sesame triple store and repository. In the file “Server\\DgsTestServer.exe.config”
please set the following values corresponding to these you set up earlier during the
installation of Sesame and the uSeekM repository:

\begin{lstlisting}[style=consola]
<appSettings>
	...
	<!-- Service Registry -->
	<add key="sesameStoreUrl" value="[Store URL,
	i.e.: http://localhost:8080/useekm-http-server/]"/>
	<add key="sesameRepository" value="[repository]"/>
	<add key="sesameUsername" value="[username]" />
	<add key="sesamePassword" value="[password]" />
</appSettings>

\end{lstlisting}

After you have done this just run the DgsTestServer.exe with administrative privileges. Now
your OMELETTE Mashup Registry is ready for usage.

\section{User manual}
The OMR provides a REST interface which is illustrated in this section. The following usage scenarios will help you get started.
Create a new resource for storing semantic component descriptions:

\begin{lstlisting}[style=consola]
POST /omr HTTP/1.1
Host: datagridservice.example.org
Content-Type: text/xml
Content-Length: xxx
<collection xmlns="http://www.w3.org/2007/app"
xmlns:atom="http://www.w3.org/2005/Atom"
xmlns:dgs="http://www.webcomposition.net/2008/02/dgs/">
	<atom:title>components</atom:title>
	<dgs:dataspaceengines>
	<dgs:dataspaceengine
	dgs:type="http://www.webcomposition.net/2008/02/dgs/DataSpaceEngines/ServiceRegistr
	yDataSpaceEngine" />
	</dgs:dataspaceengines>
</collection>

\end{lstlisting}

Add new data to the OMR:

\begin{lstlisting}[style=consola]
POST /omr/components HTTP/1.1
Host: datagridservice.example.org
Content-Type: text/xml
Content-Length: xxx
<rdf:RDF
xml:base="https://datagridservice.example.org/omr/components/graphs/1705f351-db6e-4
037-899c-08156ab31e13" xmlns:rdfs="http://www.w3.org/2000/01/rdf-schema#"
xmlns:xsd="http://www.w3.org/2001/XMLSchema#"
xmlns:ns0="http://www.ict-omelette.eu/schema.rdf#"
xmlns:rdf="http://www.w3.org/1999/02/22-rdf-syntax-ns#">
	<ns0:Widget rdf:about="http://eco.netvibes.com/themes/368491/wasabi">
		<ns1:description xmlns:ns1="http://purl.org/dc/elements/1.1/">Official theme for
		Netvibes Wasabi</ns1:description>
		<ns2:source xmlns:ns2="http://purl.org/dc/elements/1.1/"
		rdf:resource="http://eco.netvibes.com/themes/368491/wasabi" />
		<ns3:isPartOf xmlns:ns3="http://purl.org/dc/terms/"
		rdf:resource="https://datagridservice.example.org/omr/components/graphs/1705f351-db6
		e-4037-899c-08156ab31e13" />
		<ns0:categorizedBy>Textures; Official</ns0:categorizedBy>
		<ns0:endpoint
		rdf:resource="http://widgets.opera.com/widget/download/force/10322/1.0/" />
		<ns0:hasRegistryEntry
		rdf:resource="http://datagridservice.example.org/omr/components/feedItem?
		name=https://datagridservice.example.org/omr/components/graphs/1705f351-db6e-4037-
		899c-08156ab31e13" />
		<ns4:installs xmlns:ns4="http://www.netvibes.com/#">29678</ns4:installs>
		<ns5:regions xmlns:ns5="http://www.netvibes.com/#"></ns5:regions>
		<rdfs:label>Wasabi</rdfs:label>
		</ns0:Widget>
</rdf:RDF>
\end{lstlisting}
Get all available components registered in OMR:


\begin{lstlisting}[style=consola]
GET /omr/components HTTP/1.1
Host: datagridservice.example.org
Accept: text/xml

<feed xmlns="http://www.w3.org/2005/Atom">
	<title type="text">Available Components</title>
	<subtitle type="text">Component descriptions</subtitle>
	<id>http://datagridservice.example.org/omr/components</id>
	<updated>2012-05-30T15:42:26+02:00</updated>
	<link rel="meta" type="application/rdf+xml" title="Feed metadata"
	href="http://datagridservice.example.org/omr/components/meta" />
	<link rel="views" type="application/atom+xml" title="Views on RDF dataset"
	href="http://datagridservice.example.org/omr/components/views" />
	<link rel="sparql" type="application/sparql-results+xml" title="Sparql endpoint"
	href="http://datagridservice.example.org/omr/components/sparql" />
	<link rel="sparql-update" type="application/sparql-results+xml" title="Sparql-Update
	endpoint" href="http://datagridservice.example.org/omr/components/sparql-update" />
	<link rel="data" type="application/rdf+xml" title="RDF dataset"
	href="http://datagridservice.example.org/omr/components/graphs/all" />
	<link rel="search" type="application/opensearchdescription+xml" title="OpenSearch
	endpoint"
	href="http://datagridservice.example.org/omr/components/opensearch/document" />
	<entry>
	<id>https://vsr-web.informatik.tu-chemnitz.de/omr/components/graphs/bf5cdceb-38c7-4a
	0c-b5ed-49f1eae6bfa2</id>
	<title type="text">Wasabi</title>
	<summary type="text">Official theme for Netvibes Wasabi</summary>
	<published>2012-05-30T15:42:27+02:00</published>
	<updated>2012-05-30T15:42:27+02:00</updated>
	<link rel="edit-media" type="application/rdf+xml"
	href="http://datagridservice.example.org/omr/components/graphs/?
	name=https://vsr-web.informatik.tu-chemnitz.de/omr/components/graphs/bf5cdceb-38c7-4
	a0c-b5ed-49f1eae6bfa2" />
	<link rel="edit" type="application/atom+xml;type=entry;"
	href="http://datagridservice.example.org/omr/components/feedItem?
	name=https://vsr-web.informatik.tu-chemnitz.de/omr/components/graphs/bf5cdceb-38c7-4
	a0c-b5ed-49f1eae6bfa2" />
	<content type="application/rdf+xml"
	src="http://datagridservice.example.org/omr/components/graphs/?
	name=https://vsr-web.informatik.tu-chemnitz.de/omr/components/graphs/bf5cdceb-38c7-4
	a0c-b5ed-49f1eae6bfa2" />
	</entry>
	...
</feed>

\end{lstlisting}

Creation of example view for searching components by their description (SPARQL template):

\begin{lstlisting}[style=consola]
POST /omr/components/views HTTP/1.1
Host: datagridservice.example.org
Content-Type: text/xml
Content-Length: xxx
<?xml version="1.0" encoding="utf-8"?>
<entry xmlns="http://www.w3.org/2005/Atom">
	<title type="text">services</title>
	<content type="application/xml+vnd.omr">
		<omr:view xmlns:omr="http://www.ict-omelette.eu/schema.rdf#omr">
		<omr:url>?query={query=}</omr:url>
		<omr:sparql>
			<![CDATA[
			PREFIX pdc:<http://purl.org/dc/elements/1.1/>
			PREFIX search:<http://rdf.opensahara.com/search#>
			SELECT DISTINCT ?result
			WHERE
			{
			?result pdc:description ?description.
			FILTER search:text(?description, "{query}")
			}
			]]>
		</omr:sparql>
		</omr:view>
	</content>
</entry>

\end{lstlisting}

Execute view to get all components matching a given query string:

\begin{lstlisting}[style=consola]
GET /omr/components/views/services/data?query=geo HTTP/1.1
Host: datagridservice.example.org
Accept: application/sparql-results+xml
<?xml version="1.0" encoding="utf-8"?>
<sparql xmlns="http://www.w3.org/2005/sparql-results#">
	<head>
	<variable name="result" />
	</head>
	<results>
	<result>
	<binding name="result">
	<uri>http://geoservice.example.org</uri>
	</binding>
	</result>
	</results>
</sparql>

\end{lstlisting}