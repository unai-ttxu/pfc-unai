\cleardoublepage
\chapter{Introduction}

\begin{chapterintro}

This chapters provides an introduction to the problem which will be approached in this project. It provides an overview of the multi-platform application development technologies. Furthermore, a deeper description of the project and its environment is also given.

\end{chapterintro}

\cleardoublepage
\section{Context}

In the last years, mobile devices ecosystem have experimented a huge transformation. These devices have evolved improving their characteristics and changing the way we use them.

Although some of these new features, like an improved performance, have made mobile application development easier, the device market suffers a huge fragmentation which causes lots of problems when we have to deal with multi-platform mobile application developments. This phenomenon, fragmentation, occurs when some mobile users are running different operating systems or different versions of the same operating system (software fragmentation) or when some mobile users are using older devices with less powerful characteristics (hardware fragmentation).

On one hand, the coexistence of different mobile OS such as Android, iOs or Windows Phone and on the other hand the wide range of screen sizes and resolutions make multi-platform mobile application development very complicated. This problem is presented greatly within consulting companies, whose clients requires multi-platform and multi-devices application developments.

Is order to find a solution, reducing development effort and costs, we need to use multi-platform development tools that allow developers to eliminate the immense effort required to build one mobile applications for each mobile SO we need.

Within the game application development context for multi-platform devices, there are some tools such as Unity3D, Unreal Engine, Marmalade, Autodesk or Corona SDK.

Between all these development tools, Unity3D stands out due its soft learning curve, the possibility of development for multiple platforms in the same project and its huge community.

Moreover, Unity3D facilitate the management of some critical aspects within videogame development such as images, audios, shadows, physics, network and animations. Furthermore, Unity3D Asset Store, which is driven by Unity3D community, include community plugins which give us an added value.

The proposal of this project is to exemplify the development of a multi-platform mobile musical training software for children using the framework Unity3D Engine 

\section{Master thesis description}

The aim of this master thesis is the development of a multi-platform mobile musical training software for children using the framework Unity3D Engine.

This development has the purpose of training children musical skills. Through physical instrument miniatures which represent the five instrument families (percussion, keyboards, strings, woodwind, brass) the gamer will be able to interact with the application software in three different forms.

Firstly, the gamer will have the possibility to play one instrument of the selected instrument family. Secondly, the gamer will be able to interact with a whole orchestra in order to enable or disable the instruments which will be playing a classical piece. Finally, the gamer will have the possibility to read the history and characteristics of each instrument family.

The multi-platform game development includes all the following development stages:
\begin{itemize}
\item \textit{Requirement analysis}, in order to obtain the requisites given by the client for the game application.
\item \textit{Architecture design}, which represents the \textit{keyboards} instrument family, shown in Figure \ref{fig:keyboardspiece}.
\item \textit{Violin}, which represents the \textit{strings} instrument family, shown in Figure \ref{fig:stringspiece}.
\item \textit{Flute}, which represents the \textit{woodwind} instrument family, shown in Figure \ref{fig:woodwindpiece}.
\item \textit{Trumpet}, which represents the \textit{brass} instrument family, shown in Figure \ref{fig:brasspiece}.
\end{itemize}


\begin{description}

\item[OMELETTE mash-up Registry]  is the element that registers components for their usage by the rest of elements in the OMELETTE platform. A component can be a mash-up, a service, or a widget. Component descriptions and accompanying binary content will be stored in the OMR so that other elements from the OMELETTE architecture can query the OMR for them and use them. To describe components, a unified RDF component model has been defined. This unified component model provides a unified way to query components and identify, e.g., appropriate widgets for composition, interesting new services to be used when creating a widget, or relevant mash-ups of a particular domain.

\item[Automatic Discoverer] is the system responsible for populating the OMR with up-to-date components. In the current Web plenty of services and widgets are released every day, and developers need to be aware of these services in order to build state-of-the-art mash-ups. To achieve this, a module that crawls service and widget repositories registries these components into the OMR. This automatic discoverer produces semantic descriptions of the components found in the Web out of the unstructured HTML documents they are contained into.

\item[OMR admin interface] will allow to organize the components fetched by the automatic discoverer. Some of the components inserted into the OMR might be undesirable, and the administrator user within the admin interface will be responsible filter them.

\item[Web developer interface] will be used by final developer users allowing them to query by needs and recommending other services by using semantic search technologies.

\end{description}

\section{Master thesis goals}
\label{sec:masterthesisgoals}


The main purpose of this master thesis is to have a \textbf{repository} of widgets and services to build new mash-ups. First we need to \textbf{feed the repository} and we have to do it automatically using discovery techniques. The content fetched from the internet must be \textbf{structured} before it is inserted into the repository and therefore the repository has to be able to store the data with the same structure, this is done using \textbf{semantic repositories}.

All the information stored in the repository has to be \textbf{managed} manually by an administrator. This is the main goal of this master thesis, create an administration interface that permits an administrator user chose which widgets and services automatically fetched and stored are useful. The administrator interface will provide several tools to the administrator user and will use \textbf{ranking} algorithms to help him decide which ones are usefull.

Finally, there is a web interface for final users that will allow them find services and services using semantic search technologies.


\section{Structure of this Master Thesis}

In this section we will provide a brief overview of all the chapters of this Master Thesis. It
has been structured as follows:

\textit{Chapter 1} provides an introduction to the problem which will be approached in this project. It provides an overview of the benefits of mash-ups and linked data technologies. Furthermore, a deeper description of the project and its environment is also given.

\textit{Chapter 2} contains an overview of the existing technologies on which the development of the project will rely.

\textit{Chapter 3} describes one of the most important stages in software development: the requirement analysis using different scenarios. For this, a detailed analysis of the possible use cases is made using the Unified Modeling Language (UML). This language allows us to specify, build and document a system using graphic language.
The result of this evaluation will be a complete specification of the requirements, which will be matched by each module in the design stage. This helps us also to focus on key aspects and take apart other less important functionalities that could be implemented in future works.

\textit{Chapter 4} describes the architecture of the system, dividing it into 3 groups and differencing front-end and back-end modules.

\textit{Chapter 5} describes a selected use case. It is going to be explained the running of all the tools involved and its purpose. It is based on how to crawl the web to find new mash-ups, then feed the repository, do the validation and rejections of the mash-ups, and finally the developer will be able to use the discovered services.

\textit{Chapter 6} sums up the findings and conclusions found throughout the document and gives a hint about future development to continue the work done for this master thesis.

Finally, the appendix provide useful related information, especially covering the installation and configuration of the tools used in this thesis.