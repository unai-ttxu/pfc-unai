\cleardoublepage
\chapter{Introduction}

\begin{chapterintro}

This chapters provides an introduction to the problem which will be approached in this project. It provides an overview of the multi-platform application development technologies. Furthermore, a deeper description of the project and its environment is also given.

\end{chapterintro}

\cleardoublepage
\section{Context}

In the last years, mobile devices ecosystem have experimented a huge transformation. These devices have evolved improving their characteristics and changing the way we use them.

Although some of these new features, like an improved performance, have made mobile application development easier, the device market suffers a huge fragmentation which causes lots of problems when we have to deal with multi-platform mobile application developments. This phenomenon, fragmentation, occurs when some mobile users are running different operating systems or different versions of the same operating system (software fragmentation) or when some mobile users are using older devices with less powerful characteristics (hardware fragmentation).

On one hand, the coexistence of different mobile OS such as Android, iOs or Windows Phone and on the other hand the wide range of screen sizes and resolutions make multi-platform mobile application development very complicated. This problem is presented greatly within consulting companies, whose clients requires multi-platform and multi-devices application developments.

When developing native applications, developers implement an application for one specific target platform using its software development kit (SDK) and frameworks. The app is tied to that specific environment. For example, applications for Android are typically programmed in Java, access the platform functionality through Android’s frameworks, and render its user interface by employing platform-provided elements. In contrast, applications for iOS use the program- ming language Objective-C and Apple’s frameworks.

In case multiple platforms are to be supported by native applications, they have to be developed separately for each platform. This approach is the opposite of the cross-platform idea.~\cite{heneval2012}

In order to find a solution, reducing development effort and costs, we need to use multi-platform development tools that allow developers to eliminate the immense effort required to build one mobile applications for each mobile SO we need.

Cross-platform development approaches emerged to address this challenge by allowing developers to implement their apps in one step for a range of platforms, avoiding repetition and increasing productivity.~\cite{heneval2012}

Usually, targeting more than one platform normally requires developing a corresponding application for each mobile operating system. However, this approach means that the development time and hence the cost of the product will increase. A cross-platform approach, on the other hand, helps to solve this problem by developing a single code base that supports multiple platforms. Another benefit of cross-platform development, is that they allow changes to be made faster to portable mobile applications, as only one single code base needs to be modified.~\cite{chaleval2013}

Within the game application development context for multi-platform devices, there are some tools such as Unity3D, Unreal Engine, Marmalade, Autodesk or Corona SDK.

Between all these development tools, Unity3D stands out due its soft learning curve, the possibility of development for multiple platforms in the same project and its huge community.

Moreover, the functions that Unity3D supports autonomously are very abundant. In fact, all game developments are possible such as shader, physics engine, network, terrain manipulation, audio, video, and animation, and it considered so that the revision is possible to the taste of user according to the need.

Unity3D that produces based on Java script and C\# can apply and manage after producing the desired functions with script, not producing all of the programing at once. GUI composed on screen helps the first-time developer to approach easily, and the script and program that programer made with simple mouse drag.~\cite{kim2014dev}

Furthermore, Unity3D Asset Store, which is driven by Unity3D community, includes community plugins which give us an added value.

The proposal of this project is to exemplify the development of a multi-platform mobile musical training software for children using the framework Unity3D Engine.

\section{Master thesis description}

The aim of this master thesis is the development of a multi-platform mobile musical training software for children using the framework Unity3D Engine.

This development has the purpose of training children musical skills. Through physical instrument miniatures which represent the five instrument families (percussion, keyboards, strings, woodwind, brass) the gamer will be able to interact with the application software in three different forms.

Firstly, the gamer will have the possibility to play one instrument of the selected instrument family. Secondly, the gamer will be able to interact with a whole orchestra in order to enable or disable the instruments which will be playing a classical piece. Finally, the gamer will have the possibility to read the history and characteristics of each instrument family.

The multi-platform game development includes all the following development stages:
\begin{itemize}
\item \textit{Requirement analysis}, determining the needs or conditions to meet for the game application.
\item \textit{Architecture design}, defining a structured solution that meets all of the application requirements.
\item \textit{Physical instrument pieces design}, creating the physical pieces that will be used to interact with the application.
\item \textit{Software implementation}, building the multi-platform game using Unity3D engine.
\item \textit{Software test}, testing the application to check if it accomplish the acceptance criteria.
\item \textit{Application deployment}, deploying the application to the needed application markets.
\end{itemize}

Within application architecture we can distinguish the following modules:

\begin{description}

\item[Unity3D engine] is a cross-platform game engine developed by Unity Technologies and used to develop video games for PC, consoles, mobile devices and websites. Unity is notable for its ability to target games to multiple platforms. Within a project, developers have control over delivery to mobile devices, web browsers, desktops, and consoles. Supported platforms include Android, Apple TV, BlackBerry 10, iOS, Linux, Nintendo 3DS line, macOS, PlayStation 4, PlayStation Vita, Unity Web Player (including Facebook), Wii, Wii U, Windows Phone 8, Windows, Xbox 360, and Xbox One.~\cite{unitypress2}

\item[Unity Asset Store] is where a growing library of free and commercial assets are placed. These assets are created both by Unity Technologies and also members of the community. A wide variety of assets is available, covering everything from textures, models and animations to whole project examples, tutorials and Editor extensions. These assets are accessed from a simple interface built into the Unity Editor and are downloaded and imported directly into your project.

\item[Flurry Analytics] enable users to analyze consumer behavior through data observations. The platform provides features for user segmentation, consumer funnels, and applications portfolio analysis. The platform's funnels measure customized consumer conversions and trending metrics, while the portfolio analytics feature allows companies to manage entire portfolios of mobile applications with the ability to monitor data about overlap among applications as well as up-sell and cross-sell conversions.~\cite{flurry1}

\end{description}

\section{Master thesis goals}
\label{sec:masterthesisgoals}

The main purpose of this master thesis is to build a multi-platform mobile musical training software for children using the framework Unity3D Engine.

This process includes the \textbf{requirement analysis} and the \textbf{architecture design}. Due to the need of physical pieces to interact with the application the build process also include the \textbf{physical instrument pieces design}. After building the game application it has to go through previously designed \textbf{software tests} to check it fits the requirements obtained in the previous stages.

Finally, the application should be deployed to Android and iOs application markets to let users to download and use it, as long as the development team and the client would retrieve metrics and other information about the application use.


\section{Structure of this Master Thesis}

In this section we will provide a brief overview of all the chapters of this Master Thesis. It
has been structured as follows:

\textit{Chapter 1} provides an introduction to the problem which will be approached in this project. It provides an overview of the benefits of Unity3D engine. Furthermore, a deeper description of the project and its environment is also given.

\textit{Chapter 2} contains an overview of the existing technologies on which the development of the project will rely.

\textit{Chapter 3} describes one of the most important stages in software development: the requirement analysis using different scenarios. For this, a detailed analysis of the possible use cases is made using the Unified Modeling Language (UML). This language allows us to specify, build and document a system using graphic language.
The result of this evaluation will be a complete specification of the requirements, which will be matched by each module in the design stage. This helps us also to focus on key aspects and take apart other less important functionalities that could be implemented in future works.

\textit{Chapter 4} describes the architecture of the system, dividing it into two groups and differencing application software development and physical instrument pieces design.

\textit{Chapter 5} describes selected use cases. It is going to be explained the interaction with the whole game going through two of its game modes.

\textit{Chapter 6} sums up the findings and conclusions found throughout the document and gives a hint about the work done for this master thesis.

Finally, the appendix provide useful related information, especially covering the application screens.
