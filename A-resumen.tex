\cleardoublepage
\phantomsection
\chapter*{Resumen}
\addcontentsline{toc}{chapter}{Resumen}

Esta memoria es el resultado de un proyecto cuyo objetivo ha sido realizar un repositorio semántico de servicios y widgets.

Dicho repositorio tiene la facultad de poder ser rellenado de contenido de forma automática gracias a técnicas de descubrimiento automático en Internet, haciendo uso de arañas o scrappers que recogen el contenido y posteriormente lo convierten a formato estructurado o semántico.

Para que dicho repositorio tenga contenido de utilidad se han presentado algoritmos que son capaces de calificar los servicios y además se ha desarrollado una herramienta que permite organizar y administrar todo el contenido extraído por las herramientas de descubrimiento automático.

Posteriormente se ha presentado otra herramienta que permite al usuario final explorar el repositorio de servicios y ayudarle a buscar y a encontrar el contenido deseado mediante técnicas de búsqueda semántica.

Por último, se han presentado las conclusiones extraídas del trabajo, las posibles líneas de continuación del proyecto, así como los siguientes pasos en cuanto a desarrollo y aprovechamiento de la plataforma.

\vfill
\textbf{Palabras clave:} Tecnologías semánticas, Linked data, OpenRDF Sesame, Linked Media Framework, RDF, SPARQL, PHP, JavaScript, Java, Knowckout JS