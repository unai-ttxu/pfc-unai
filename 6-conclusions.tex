\chapter{Conclusions and future lines}
\label{chap:conclusions}
\begin{chapterintro}
In this chapter we will describe the conclusions extracted from this master thesis
\end{chapterintro}

\cleardoublepage
\section{Conclusions}

By using Unity3D as the base to build the application we have been able to create a crossplatform mobile game without the effort requires to manage separate developments for each platform.



\section{Achieved goals}

In this section we will detail our application's achieved goals checking if we have covered all the use cases presented in section \ref{subsec:gamemodes} 

\begin{description}
\item[Play an instrument]
This goal has been achieved successfully. This use case is described in \ref{subsec:playinstrument}. The gamer is able to play five different instruments, one of each musical instrument family. In order to select the instrument to play with, the gamer is able to place the physical miniature, which represents the instrument family, on a recognition zone in the screen. Also, the gamer can watch a demo or change the melody to be played. We can see this achievements in figures \ref{fig:playing_xylo_start_screen}, \ref{fig:playing_piano_screen}, \ref{fig:playing_harp_screen}, \ref{fig:playing_panpipes_screen} and \ref{fig:playing_trombone_screen}

\item[Conduct the orchestra]
This goal has been achieved successfully. This use case is described in \ref{subsec:conductorchestra}. The gamer is able to conduct the orchestra that is playing the selected melody. The gamer can conduct the melody by enabling or disabling the instruments that are playing this melody. In order to select the instrument family whose instruments will be able to be enabled or disabled, the gamer is able to place the physical miniature, which represents the instrument family, on a recognition zone in the screen. Also, the gamer can watch stop or change the melody to be conducted. We can see this achievements in figures \ref{fig:conducting_all_stop_screen} and \ref{fig:conducting_some_screen}.

\item[Discover an instrument]
This goal has been achieved successfully. This use case is described in \ref{subsec:discoverinstrument}. The gamer is able to read information of instruments of the five different musical families. In order to select the instruments to discover, the gamer is able to place the physical miniature, which represents the instrument family, on a recognition zone in the screen. Also the gamer can reproduce the instrument selected sound. We can see this achievements in figures shown in section \ref{sec:discoveringscreens}

\item[Watch a melody play demo]
This goal has been achieved successfully. This use case is described in \ref{subsec:watchdemo}. Gamer is able to watch a demo of all available melodies in the \textit{Play instrument} game mode with each of the five instruments the gamer is able to play with.

\item[Select a melody]
This goal has been achieved successfully. This use case is described in \ref{subsec:selectmelody}. Gamer is able to select a melody within both \textit{Play instrument} and \textit{Conduct orchestra} game modes.  We can see this achievements in figures \ref{fig:melodies_playing_screen} and \ref{fig:conducting_melodies_screen}.

\item[Select an instrument]
This goal has been achieved successfully. This use case is described in \ref{subsec:selectinstrument}. As we have said in the previous achievements, the gamer is able to choose an instrument within all game modes. In order to select the instrument, the gamer is able to place the physical miniature, which represents the instrument family, on a recognition zone in the screen.

\end{description}